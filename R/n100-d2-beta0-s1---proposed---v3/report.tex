% Options for packages loaded elsewhere
\PassOptionsToPackage{unicode}{hyperref}
\PassOptionsToPackage{hyphens}{url}
\documentclass[
]{article}
\usepackage{xcolor}
\usepackage[margin=1in]{geometry}
\usepackage{amsmath,amssymb}
\setcounter{secnumdepth}{-\maxdimen} % remove section numbering
\usepackage{iftex}
\ifPDFTeX
  \usepackage[T1]{fontenc}
  \usepackage[utf8]{inputenc}
  \usepackage{textcomp} % provide euro and other symbols
\else % if luatex or xetex
  \usepackage{unicode-math} % this also loads fontspec
  \defaultfontfeatures{Scale=MatchLowercase}
  \defaultfontfeatures[\rmfamily]{Ligatures=TeX,Scale=1}
\fi
\usepackage{lmodern}
\ifPDFTeX\else
  % xetex/luatex font selection
\fi
% Use upquote if available, for straight quotes in verbatim environments
\IfFileExists{upquote.sty}{\usepackage{upquote}}{}
\IfFileExists{microtype.sty}{% use microtype if available
  \usepackage[]{microtype}
  \UseMicrotypeSet[protrusion]{basicmath} % disable protrusion for tt fonts
}{}
\makeatletter
\@ifundefined{KOMAClassName}{% if non-KOMA class
  \IfFileExists{parskip.sty}{%
    \usepackage{parskip}
  }{% else
    \setlength{\parindent}{0pt}
    \setlength{\parskip}{6pt plus 2pt minus 1pt}}
}{% if KOMA class
  \KOMAoptions{parskip=half}}
\makeatother
\usepackage{longtable,booktabs,array}
\usepackage{calc} % for calculating minipage widths
% Correct order of tables after \paragraph or \subparagraph
\usepackage{etoolbox}
\makeatletter
\patchcmd\longtable{\par}{\if@noskipsec\mbox{}\fi\par}{}{}
\makeatother
% Allow footnotes in longtable head/foot
\IfFileExists{footnotehyper.sty}{\usepackage{footnotehyper}}{\usepackage{footnote}}
\makesavenoteenv{longtable}
\usepackage{graphicx}
\makeatletter
\newsavebox\pandoc@box
\newcommand*\pandocbounded[1]{% scales image to fit in text height/width
  \sbox\pandoc@box{#1}%
  \Gscale@div\@tempa{\textheight}{\dimexpr\ht\pandoc@box+\dp\pandoc@box\relax}%
  \Gscale@div\@tempb{\linewidth}{\wd\pandoc@box}%
  \ifdim\@tempb\p@<\@tempa\p@\let\@tempa\@tempb\fi% select the smaller of both
  \ifdim\@tempa\p@<\p@\scalebox{\@tempa}{\usebox\pandoc@box}%
  \else\usebox{\pandoc@box}%
  \fi%
}
% Set default figure placement to htbp
\def\fps@figure{htbp}
\makeatother
\setlength{\emergencystretch}{3em} % prevent overfull lines
\providecommand{\tightlist}{%
  \setlength{\itemsep}{0pt}\setlength{\parskip}{0pt}}
\usepackage{bookmark}
\IfFileExists{xurl.sty}{\usepackage{xurl}}{} % add URL line breaks if available
\urlstyle{same}
\hypersetup{
  pdftitle={Simulation Results},
  hidelinks,
  pdfcreator={LaTeX via pandoc}}

\title{Simulation Results}
\author{}
\date{\vspace{-2.5em}2026-02-02}

\begin{document}
\maketitle

\subsection{Simulation Setup}\label{simulation-setup}

This simulation is performed with \(n=100\) and \(d=2\), using the 2-d
lattice as the underlying graph. \(s=1\) parameters are set to be
nonzero, and the beta parameter is chosen to be \(\beta=0\). The
attached results are for a 2-replication simulation. The parameter
vector \(\theta\) has sparse components other than the following:

\begin{table}[!h]
\centering
\begin{tabular}{r|r}
\hline
Parameter.Index & Value\\
\hline
1 & -1\\
\hline
\end{tabular}
\end{table}

but for brevity, our simulation only estimates the indices of \(\theta\)
in \(\mathcal C =\{\) 1, 1\(\}\) elements of \(\theta\). Accordingly,
\textbf{all statistics and visuals are indicative of performance only on
the set \(\mathcal C\).}

The results from our code are not augmented with any comparison method
here.

The attached results include the mean-squared error for each parameter
estimate, as well as boxplots for a selection of nonzero and zero-valued
parameters. In the boxplots, the green line represents the true value of
the estimated parameter.

After these, I show coverage statistics for 95\% symmetric confidence
intervals for each of the parameters.

\subsection{Results}\label{results}

\subsubsection{Mean-squared error
comparison}\label{mean-squared-error-comparison}

\begin{longtable}[]{@{}lr@{}}
\caption{Mean-Squared Error of Parameter Estimates}\tabularnewline
\toprule\noalign{}
& proposed \\
\midrule\noalign{}
\endfirsthead
\toprule\noalign{}
& proposed \\
\midrule\noalign{}
\endhead
\bottomrule\noalign{}
\endlastfoot
theta{[}1{]} & 0.236 \\
theta{[}1{]} & 0.236 \\
total & 0.236 \\
\end{longtable}

\begin{longtable}[]{@{}lr@{}}
\caption{Mean-Squared Error of First-Step Parameter
Estimates}\tabularnewline
\toprule\noalign{}
& proposed \\
\midrule\noalign{}
\endfirsthead
\toprule\noalign{}
& proposed \\
\midrule\noalign{}
\endhead
\bottomrule\noalign{}
\endlastfoot
theta{[}1{]} & 0.842 \\
theta{[}1{]} & 0.842 \\
total & 0.842 \\
\end{longtable}

\newpage

\begin{verbatim}
### Mean absolute deviation comparison $(\frac{1}{n.sim}\sum_{i=1}^{n.sim} \frac{1}{|\mathcal C|}\|\hat\theta_{i,\mathcal C}-\theta_{\mathcal C}\|)$
\end{verbatim}

\begin{longtable}[]{@{}lr@{}}
\caption{Mean Absolute Deviation of Parameter Estimates}\tabularnewline
\toprule\noalign{}
& proposed \\
\midrule\noalign{}
\endfirsthead
\toprule\noalign{}
& proposed \\
\midrule\noalign{}
\endhead
\bottomrule\noalign{}
\endlastfoot
theta{[}1{]} & 0.441 \\
theta{[}1{]} & 0.441 \\
total & 0.441 \\
\end{longtable}

\begin{longtable}[]{@{}lr@{}}
\caption{Mean Absolute Deviation of First-Step Parameter
Estimates}\tabularnewline
\toprule\noalign{}
& proposed \\
\midrule\noalign{}
\endfirsthead
\toprule\noalign{}
& proposed \\
\midrule\noalign{}
\endhead
\bottomrule\noalign{}
\endlastfoot
theta{[}1{]} & 0.651 \\
theta{[}1{]} & 0.651 \\
total & 0.651 \\
\end{longtable}

\newpage

\subsubsection{Boxplots}\label{boxplots}

\pandocbounded{\includegraphics[keepaspectratio]{C:/Users/josmi/UFL Dropbox/Joshua Miles/Overleaf/Inference_Ising/Code/misle/R/n100-d2-beta0-s1---proposed---v3/report_files/figure-latex/unnamed-chunk-14-1.pdf}}

\newpage

\newpage

\subsubsection{Statistics and 95\% Confidence Intervals from
per-Replicate
Estimates}\label{statistics-and-95-confidence-intervals-from-per-replicate-estimates}

\vspace{5pt}

\subsubsection{Statistics for Theoretical 95\% Confidence
Intervals}\label{statistics-for-theoretical-95-confidence-intervals}

\end{document}
