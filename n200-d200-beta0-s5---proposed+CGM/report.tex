% Options for packages loaded elsewhere
\PassOptionsToPackage{unicode}{hyperref}
\PassOptionsToPackage{hyphens}{url}
\documentclass[
]{article}
\usepackage{xcolor}
\usepackage[margin=1in]{geometry}
\usepackage{amsmath,amssymb}
\setcounter{secnumdepth}{-\maxdimen} % remove section numbering
\usepackage{iftex}
\ifPDFTeX
  \usepackage[T1]{fontenc}
  \usepackage[utf8]{inputenc}
  \usepackage{textcomp} % provide euro and other symbols
\else % if luatex or xetex
  \usepackage{unicode-math} % this also loads fontspec
  \defaultfontfeatures{Scale=MatchLowercase}
  \defaultfontfeatures[\rmfamily]{Ligatures=TeX,Scale=1}
\fi
\usepackage{lmodern}
\ifPDFTeX\else
  % xetex/luatex font selection
\fi
% Use upquote if available, for straight quotes in verbatim environments
\IfFileExists{upquote.sty}{\usepackage{upquote}}{}
\IfFileExists{microtype.sty}{% use microtype if available
  \usepackage[]{microtype}
  \UseMicrotypeSet[protrusion]{basicmath} % disable protrusion for tt fonts
}{}
\makeatletter
\@ifundefined{KOMAClassName}{% if non-KOMA class
  \IfFileExists{parskip.sty}{%
    \usepackage{parskip}
  }{% else
    \setlength{\parindent}{0pt}
    \setlength{\parskip}{6pt plus 2pt minus 1pt}}
}{% if KOMA class
  \KOMAoptions{parskip=half}}
\makeatother
\usepackage{longtable,booktabs,array}
\usepackage{calc} % for calculating minipage widths
% Correct order of tables after \paragraph or \subparagraph
\usepackage{etoolbox}
\makeatletter
\patchcmd\longtable{\par}{\if@noskipsec\mbox{}\fi\par}{}{}
\makeatother
% Allow footnotes in longtable head/foot
\IfFileExists{footnotehyper.sty}{\usepackage{footnotehyper}}{\usepackage{footnote}}
\makesavenoteenv{longtable}
\usepackage{graphicx}
\makeatletter
\newsavebox\pandoc@box
\newcommand*\pandocbounded[1]{% scales image to fit in text height/width
  \sbox\pandoc@box{#1}%
  \Gscale@div\@tempa{\textheight}{\dimexpr\ht\pandoc@box+\dp\pandoc@box\relax}%
  \Gscale@div\@tempb{\linewidth}{\wd\pandoc@box}%
  \ifdim\@tempb\p@<\@tempa\p@\let\@tempa\@tempb\fi% select the smaller of both
  \ifdim\@tempa\p@<\p@\scalebox{\@tempa}{\usebox\pandoc@box}%
  \else\usebox{\pandoc@box}%
  \fi%
}
% Set default figure placement to htbp
\def\fps@figure{htbp}
\makeatother
\setlength{\emergencystretch}{3em} % prevent overfull lines
\providecommand{\tightlist}{%
  \setlength{\itemsep}{0pt}\setlength{\parskip}{0pt}}
\usepackage{bookmark}
\IfFileExists{xurl.sty}{\usepackage{xurl}}{} % add URL line breaks if available
\urlstyle{same}
\hypersetup{
  pdftitle={Simulation Results},
  hidelinks,
  pdfcreator={LaTeX via pandoc}}

\title{Simulation Results}
\author{}
\date{\vspace{-2.5em}2026-01-20}

\begin{document}
\maketitle

\subsection{Simulation Setup}\label{simulation-setup}

This simulation is performed with \(n=200\) and \(d=200\), using the 2-d
lattice as the underlying graph. \(s=5\) parameters are set to be
nonzero, and the beta parameter is chosen to be \(\beta=0\). The
attached results are for a 10-replication simulation. The parameter
vector \(\theta\) has sparse components other than the following:

\begin{verbatim}
$\theta$$[$2$] = -0.447$\theta$$[$50$] = 0.447$\theta$$[$67$] = 0.447$\theta$$[$92$] = 0.447$\theta$$[$104$] = 0.447
\end{verbatim}

but for brevity, our simulation only estimates the indices of \(\theta\)
in \(\mathcal C =\{\) 2, 50, 71, 172\(\}\) elements of \(\theta\).
Accordingly, \textbf{all statistics and visuals are indicative of
performance only on the set \(\mathcal C\).}

The results from our code are compared to those of
\href{http://www-stat.wharton.upenn.edu/~tcai/paper/Inference-GLM.pdf}{Cai,
Guo, and Ma (2021)}.

The attached results include the mean-squared error for each parameter
estimate, as well as boxplots for a selection of nonzero and zero-valued
parameters. In the boxplots, the green line represents the true value of
the estimated parameter.

After these, I show coverage statistics for 95\% symmetric confidence
intervals for each of the parameters.

\subsection{Results}\label{results}

\begin{verbatim}
### Mean-squared error comparison $(\frac{1}{n.sim}\sum_{i=1}^{n.sim} \frac{1}{|\mathcal C|}\|\hat\theta_{i,\mathcal C}-\theta_{\mathcal C}\|^2)$
\end{verbatim}

\begin{longtable}[]{@{}lrr@{}}
\caption{Mean-Squared Error of Parameter Estimates}\tabularnewline
\toprule\noalign{}
& proposed & cgm \\
\midrule\noalign{}
\endfirsthead
\toprule\noalign{}
& proposed & cgm \\
\midrule\noalign{}
\endhead
\bottomrule\noalign{}
\endlastfoot
theta{[}2{]} & 0.070 & 2.048 \\
theta{[}50{]} & 0.062 & 0.514 \\
theta{[}71{]} & 0.007 & 0.366 \\
theta{[}172{]} & 0.005 & 0.050 \\
total & 0.036 & 0.745 \\
\end{longtable}

\begin{longtable}[]{@{}lrr@{}}
\caption{Mean-Squared Error of First-Step Parameter
Estimates}\tabularnewline
\toprule\noalign{}
& proposed & cgm \\
\midrule\noalign{}
\endfirsthead
\toprule\noalign{}
& proposed & cgm \\
\midrule\noalign{}
\endhead
\bottomrule\noalign{}
\endlastfoot
theta{[}2{]} & 0.171 & 0.040 \\
theta{[}50{]} & 0.200 & 0.061 \\
theta{[}71{]} & 0.000 & 0.000 \\
theta{[}172{]} & 0.000 & 0.000 \\
total & 0.093 & 0.025 \\
\end{longtable}

\newpage

\begin{verbatim}
### Mean absolute deviation comparison $(\frac{1}{n.sim}\sum_{i=1}^{n.sim} \frac{1}{|\mathcal C|}\|\hat\theta_{i,\mathcal C}-\theta_{\mathcal C}\|)$
\end{verbatim}

\begin{longtable}[]{@{}lrr@{}}
\caption{Mean Absolute Deviation of Parameter Estimates}\tabularnewline
\toprule\noalign{}
& proposed & cgm \\
\midrule\noalign{}
\endfirsthead
\toprule\noalign{}
& proposed & cgm \\
\midrule\noalign{}
\endhead
\bottomrule\noalign{}
\endlastfoot
theta{[}2{]} & 0.254 & 0.579 \\
theta{[}50{]} & 0.224 & 0.504 \\
theta{[}71{]} & 0.055 & 0.360 \\
theta{[}172{]} & 0.056 & 0.165 \\
total & 0.147 & 0.402 \\
\end{longtable}

\begin{longtable}[]{@{}lrr@{}}
\caption{Mean Absolute Deviation of First-Step Parameter
Estimates}\tabularnewline
\toprule\noalign{}
& proposed & cgm \\
\midrule\noalign{}
\endfirsthead
\toprule\noalign{}
& proposed & cgm \\
\midrule\noalign{}
\endhead
\bottomrule\noalign{}
\endlastfoot
theta{[}2{]} & 0.407 & 0.154 \\
theta{[}50{]} & 0.447 & 0.215 \\
theta{[}71{]} & 0.000 & 0.005 \\
theta{[}172{]} & 0.000 & 0.000 \\
total & 0.214 & 0.094 \\
\end{longtable}

\newpage

\subsubsection{Boxplots}\label{boxplots}

\pandocbounded{\includegraphics[keepaspectratio]{C:/Users/josmi/UFL Dropbox/Joshua Miles/Overleaf/Inference_Ising/Code/misle/n200-d200-beta0-s5---proposed+CGM/report_files/figure-latex/unnamed-chunk-12-1.pdf}}
\pandocbounded{\includegraphics[keepaspectratio]{C:/Users/josmi/UFL Dropbox/Joshua Miles/Overleaf/Inference_Ising/Code/misle/n200-d200-beta0-s5---proposed+CGM/report_files/figure-latex/unnamed-chunk-12-2.pdf}}
\pandocbounded{\includegraphics[keepaspectratio]{C:/Users/josmi/UFL Dropbox/Joshua Miles/Overleaf/Inference_Ising/Code/misle/n200-d200-beta0-s5---proposed+CGM/report_files/figure-latex/unnamed-chunk-12-3.pdf}}
\pandocbounded{\includegraphics[keepaspectratio]{C:/Users/josmi/UFL Dropbox/Joshua Miles/Overleaf/Inference_Ising/Code/misle/n200-d200-beta0-s5---proposed+CGM/report_files/figure-latex/unnamed-chunk-12-4.pdf}}

\newpage

\subsubsection{First Step Histograms}\label{first-step-histograms}

\pandocbounded{\includegraphics[keepaspectratio]{C:/Users/josmi/UFL Dropbox/Joshua Miles/Overleaf/Inference_Ising/Code/misle/n200-d200-beta0-s5---proposed+CGM/report_files/figure-latex/unnamed-chunk-14-1.pdf}}
\pandocbounded{\includegraphics[keepaspectratio]{C:/Users/josmi/UFL Dropbox/Joshua Miles/Overleaf/Inference_Ising/Code/misle/n200-d200-beta0-s5---proposed+CGM/report_files/figure-latex/unnamed-chunk-14-2.pdf}}
\pandocbounded{\includegraphics[keepaspectratio]{C:/Users/josmi/UFL Dropbox/Joshua Miles/Overleaf/Inference_Ising/Code/misle/n200-d200-beta0-s5---proposed+CGM/report_files/figure-latex/unnamed-chunk-14-3.pdf}}
\pandocbounded{\includegraphics[keepaspectratio]{C:/Users/josmi/UFL Dropbox/Joshua Miles/Overleaf/Inference_Ising/Code/misle/n200-d200-beta0-s5---proposed+CGM/report_files/figure-latex/unnamed-chunk-14-4.pdf}}

\newpage

\subsubsection{Statistics and 95\% Confidence Intervals from
per-Replicate
Estimates}\label{statistics-and-95-confidence-intervals-from-per-replicate-estimates}

\begin{longtable}[]{@{}lrrrrr@{}}
\caption{Statistics for proposed Estimates}\tabularnewline
\toprule\noalign{}
& Min & Median & Max & lower.CI.btsp & upper.CI.btsp \\
\midrule\noalign{}
\endfirsthead
\toprule\noalign{}
& Min & Median & Max & lower.CI.btsp & upper.CI.btsp \\
\midrule\noalign{}
\endhead
\bottomrule\noalign{}
\endlastfoot
theta{[}2{]} & -0.334 & -0.185 & -0.036 & -0.319 & -0.064 \\
theta{[}50{]} & 0.031 & 0.216 & 0.382 & 0.048 & 0.372 \\
theta{[}71{]} & -0.032 & -0.003 & 0.178 & -0.030 & 0.175 \\
theta{[}172{]} & -0.108 & 0.013 & 0.131 & -0.093 & 0.130 \\
\end{longtable}

\begin{longtable}[]{@{}lrrrrr@{}}
\caption{Statistics for cgm Estimates}\tabularnewline
\toprule\noalign{}
& Min & Median & Max & lower.CI.btsp & upper.CI.btsp \\
\midrule\noalign{}
\endfirsthead
\toprule\noalign{}
& Min & Median & Max & lower.CI.btsp & upper.CI.btsp \\
\midrule\noalign{}
\endhead
\bottomrule\noalign{}
\endlastfoot
theta{[}2{]} & -0.716 & -0.392 & 4.047 & -0.672 & 3.107 \\
theta{[}50{]} & -1.386 & 0.435 & 0.686 & -1.205 & 0.685 \\
theta{[}71{]} & -1.762 & 0.108 & 0.401 & -1.418 & 0.400 \\
theta{[}172{]} & -0.153 & 0.075 & 0.556 & -0.141 & 0.497 \\
\end{longtable}

\vspace{5pt}

\subsubsection{Statistics for Theoretical 95\% Confidence
Intervals}\label{statistics-for-theoretical-95-confidence-intervals}

\begin{longtable}[]{@{}lrrrrr@{}}
\caption{Theoretical 95\% Confidence Interval Statistics (averaged
across replications) for proposed Estimates}\tabularnewline
\toprule\noalign{}
& Estimate & SE & lower.CI & upper.CI & cvg \\
\midrule\noalign{}
\endfirsthead
\toprule\noalign{}
& Estimate & SE & lower.CI & upper.CI & cvg \\
\midrule\noalign{}
\endhead
\bottomrule\noalign{}
\endlastfoot
theta{[}2{]} & -0.193 & 0.103 & -0.394 & 0.008 & 0.2 \\
theta{[}50{]} & 0.224 & 0.107 & 0.013 & 0.434 & 0.4 \\
theta{[}71{]} & 0.034 & 0.101 & -0.164 & 0.232 & 1.0 \\
theta{[}172{]} & 0.020 & 0.103 & -0.182 & 0.222 & 1.0 \\
\end{longtable}

\begin{longtable}[]{@{}lrrrrr@{}}
\caption{Theoretical 95\% Confidence Interval Statistics (averaged
across replications) for cgm Estimates}\tabularnewline
\toprule\noalign{}
& Estimate & SE & lower.CI & upper.CI & cvg \\
\midrule\noalign{}
\endfirsthead
\toprule\noalign{}
& Estimate & SE & lower.CI & upper.CI & cvg \\
\midrule\noalign{}
\endhead
\bottomrule\noalign{}
\endlastfoot
theta{[}2{]} & 0.040 & 0.237 & -0.426 & 0.505 & 0.9 \\
theta{[}50{]} & 0.158 & 0.189 & -0.213 & 0.528 & 0.6 \\
theta{[}71{]} & -0.043 & 0.183 & -0.402 & 0.315 & 0.8 \\
theta{[}172{]} & 0.115 & 0.159 & -0.197 & 0.427 & 0.9 \\
\end{longtable}

\end{document}
